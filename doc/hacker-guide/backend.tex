Backend consists compilation passes of RSSA and Machine. It serves as
a middle layer between the (heavily optimized) SSA pass and lower
level code generators.

Various interestings happen at the backend part. Objects are finally
concretely represented (packed-representation). GC checks are
implemented (limit-check). Signal handlers are implemented
(implement-handlers). Live variables are detected (live). Temporary
variable allocation locations are determined
(allocate-registers). They are depicted in the following graph:

\begin{figure}
\centering
\includegraphics[width=.5\textwidth]{figures/mlton-backend.png}
\caption{MLton Backend Overview}
\label{fig:backend-overview}
\end{figure}

\subsection{Packed Representation}
Recall that at runtime, there are multiple types of objects: Normal,
Stack, Array and Weak. We do a re-cap of how arrays and normal objects
look like here. They are depicted in Figure.~\ref{fig:object-layout}.



\subsection{Limit Check}

\subsection{Implement Handlers}

\subsection{Liveness}

\subsection{Allocate Registers}
For limit checks, when the fixed size allocations that follow a variable size
allocation are stored in the ``bytesAllocated'' field of the limitCheck field in
the AllocateArray variant of machine-output.sig
